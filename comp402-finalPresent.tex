\documentclass[10pt]{article}
\usepackage{amsmath, amssymb, amsthm}
\usepackage{setspace}
\usepackage{hyperref}

% Setting Margins
\setlength{\textheight}{9in} \setlength{\topmargin}{-.5in}
\setlength{\textwidth}{6.5in} \setlength{\oddsidemargin}{0in}
\setlength{\evensidemargin}{0in}

\newcounter{enumi_saved}

\title{COMP402 - Final Presentation}
\author{}
\date{ Spring 2013}

\begin{document}
\maketitle
\thispagestyle{empty}

\begin{itemize}
\item \textbf{When:} Saturday 5/4 at 1pm
\item \textbf{Where:} TBD (HT 207 if all else fails)
\item \textbf{What:} 20-25 minute presentation + questions 
\end{itemize}

\subsubsection*{Presentation Content}

The emphasis of your presentation is not what you produced, but how you produced it.
\begin{itemize}
\item (0-5 minutes) Project Summary and Demo (the what part)
\item (15-20 minutes) Process Presentation and Reflection (the how part)
\begin{itemize}
\item Summary of your research/development process and major tasks involved in that process
\item \textit{Identification of main technical hurdles and how they were overcome}
\item \textit{Critique and Evaluation of process}. 
\item Identification of the specific element of the work of which you're most proud
\end{itemize}
\end{itemize}
The real meat of the presentation are the sections listed in italics.  In these sections you will implicitly demonstrate your understanding of effective and efficient research and development practices in computing. 
Do your actions have a basis in best-practices? What worked and what didn't and why? 

\end{document}
\documentclass[10pt]{article}
\usepackage{amsmath}
\usepackage{setspace}
\usepackage{hyperref}


\setlength{\textheight}{9in} \setlength{\topmargin}{-.5in}
\setlength{\textwidth}{6.5in} \setlength{\oddsidemargin}{0in}
\setlength{\evensidemargin}{0in}

\title{COMP 401 - Scientific, Research Based Project Proposal}
\author{  }
\date{Fall 2015}


\begin{document}
\maketitle

COMP401 culminates with the formal proposal of your senior capstone project.  The proposal consists of three pieces
\begin{enumerate}
\item Written Proposal
\item Presentation and Defense 
\item Poster
\end{enumerate}
Your presentation is a summary of the written proposal given in a public forum while the poster serves as a condensed snap shot of the project you'll be implementing in COMP402.  Neither should introduce information not found in the written proposal.

\section{Written Proposal}

Your written proposal should contain the sections listed below.  Each section has an estimated page length next to it.  Page length estimates assume \textit{single spaced, 11 pt. font}.   
\begin{enumerate}
\item Abstract [1-2]
\item Thesis/Purpose Statement [1-2]
\item Background/Introduction [3-5]
\item Importance of Your Research [5-7]
\item Users, Stakeholders, and Ethical Questions [1-5]
\item Summary of Plan of Research [1-2]
\item Tools and Resources Needed [1-2]
\item Identification and Introduction to Key Research [5-10]
\item Proposed Time Line [1-2]
\item Conclusion [1-2]
\item Bibliography \& Works Cited [1-2]
\end{enumerate}

\subsection{Abstract}
The abstract should be roughly one to two pages and should give a high level summary of the document.  One should be able to read your abstract and determine what your project is and why it's interesting/important.  You should craft your abstract such that other researchers and interested parties can quickly determine if your work is something they need or want to read.  In the context of a proposal for research, an abstract might be used by funding agencies like the National Science Foundation to determine which of the large number of proposals they received are interesting enough to make the first cut.  Be sure that your abstract hits the high points of your research and makes readers want to read the entire proposal.

\subsection{Thesis/Purpose Statement}

In this section you should focus on making the core elements of your research clear.  What question are you trying to answer? What do you expect the answer to be \textit{and why}?  Keep the focus of this section on your research and provide only enough background as is needed to justify your assertions. 


\subsection{Background/Introduction}

This section introduces the research area with which your own research is associated.  You should address:
\begin{itemize}
\item The history of the research community and the problems it aims to address
\item Current frontiers in the area of research 
\item Summary of the major gatherings and publications associated with the community 
\item The research and researchers most closely associated with your research
\end{itemize}
This section is critical for putting your research in the appropriate context.  It also demonstrates an understanding of how the work you're doing may impact the scientific work being done by others.

\subsection{Importance of Your Research}

Now that you've provided a scientific context for your research, you should highlight what makes the work you'll be doing unique or important.  Along these lines you should:
\begin{itemize}
\item Compare and contrast your work with similar research
\item Identify the unique elements of your work
\item Identify the need for your work both in the relevant research community and and wider context of computing
\end{itemize}

\subsection{Users, Stakeholders and Ethical Questions}

Science should not be practiced for science's sake alone but should be done with careful consideration for the impact it will have on society at large.  This section asks you to address the impact that your work, and the work of your research community, on different social groups and business.  Be certain to identify both positive and negative impacts.  

\subsection{Summary of Plan of Research}

Here you answer the question: ``What do you plan to do?''  Where previous sections focus on the central question, outcome, and impact of your research, this section asks you to describe exactly what you plan to do and how it helps you prove your assertion.  When applicable, this section should include discussions and details of an empirical methods used .

\subsection{Tools and Resources Needed}

In this section you should describe and list any tools and resources you'll need to complete your research.  If any of these resources are not freely available, you should discuss how you're obtaining these resources.  

\subsection{Identification and Introduction to Key Research}

At this point you'll provide clear evidence that you've done some of the initial literature research by identifying the key papers that lead up to your specific research question.  You might imagine this section as an annotated ``best of'' selection from your bibliography.  Be sure that through the course of this section you show the tread that runs through these works and ends at your work.


\subsection{Proposed  Time Line}

Give a \textbf{weekly} time line for carrying out your research over the course of the spring semester.  Be sure to budget your time for both experimentation/exploration as well as writing the final paper.  You should have clear, explicit tasks listed for each week.  \textit{In the spring, you will be doing checkpoint presentations based on this time line. It will be your responsibility to either stick to your time line or be able to give reasonable justification for deviations from your time line.  Being unprepared for weekly checkpoints will have a negative effect on your grade.  It is therefore important that you think hard about how to budget and manage your time with respect to this project. } 

\subsection{Conclusion}

The final section of your proposal should be both a summary of the proposal and a place for you to reflect on the work ahead of you.  Be certain to write about:
\begin{itemize}
\item The importance/significance of your research
\item The technical/intellectual challenges that you foresee in carrying out your research
\item The personal challenges that you foresee in carrying out your research
\end{itemize}
Make it clear to the reader that not only is this research interesting and compelling but that you'll be successful in carrying it out.  

\subsection{Bibliography \& Works Citied}

Any information that is not your own and is used in the writing of your proposal should be properly cited.  If you refer to technical details about software/hardware that you yourself did not create, then you should cite a reliable source for this information.  Beyond the citation of information your bibliography should include:
\begin{itemize}
\item At least one entry to a reliable Reference \textit{for each} item listed in your tools and resources section  
\end{itemize}
Essentially, you need to show that you have at least one reference or guide for each piece of computing technology (language, software, hardware, etc. ) that you will use to carry out your research along with any other sources that you used in researching the various aspects of your project.

\subsubsection{ Regarding Wikipedia and Web Sources}
Wikipedia is a great starting place, but rarely is it a good final source.  If you directly cite a Wikipedia entry you should reference it, but you should really seek out a primary source for that information.  \textit{Obvious reliance on Wikipedia as your primary source of information is likely to reflect negatively on your grade.  Given the prevalence of links to other sources at the end of most Wikipedia entries, failure to explore those sources in search of a primary source is just lazy.}  The above comments are limited to Wikipedia and do not generalize to all web sources.
It is highly likely that many of your references will be web-based.  Most computing technologies, be they 'hard' or 'soft', have online documentation produced by the technology's creators.  You are should seek these references out and \textit{include them in your bibliography when available.}  If there is a source of information that you plan to use to aid in the completion of your project, then it should be in your bibliography.

\section{Presentation and Defense}

You'll be giving a \textit{thirty minute} proposal of your research plan the focus of which should be on the background and justification for your central assertion. The defense aspect of the presentation comes from the fact that you should be prepared to discuss/defend the reasoning behind the research.  The goal of the proposal is to be certain that you have a clear path laid out for success.  The presentation is the time when you convince us that what you wrote is more than just words on paper.  

Here's a rough outline of what your presentation should entail with time estimates, in minutes, for each item.  The information for each section should mirror/summarize the relevant section from the written proposal. 
\begin{itemize}
\item Background/Introduction [2-5]
\item Thesis/Purpose Statement [1-2]
\item Importance of Research [2-3]
\item Key Research [5-15]
\item Users, Stakeholders, and Ethical Questions [2-3]
\item Research Plan [2-3]
\item Tools and Resources [2-3]
\item Proposed Time Line [2-3]
\item Conclusion [2-3]
\end{itemize}
You must have Power Point slides to accompany the presentation.

\section{Poster}

Your poster will act as a snapshot of the written proposal but without a majority of the technical details.  It touches on information from the following sections of your proposal.
\begin{itemize}
\item Background/Introduction
\item Thesis Statement
\item Statement of Importance
\item Users, Stakeholders, and Ethical Questions
\item Tools and Resources
\end{itemize}
The posters will be hung in the hallway where the Math/CS offices can be found.  They are intended to get people excited about your work and give them a feel for the types of problems your research addresses as well as the importance of your research. 



\end{document}
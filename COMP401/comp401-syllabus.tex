\documentclass[10pt]{article}
\usepackage{amsmath}
\usepackage{setspace}
\usepackage{hyperref}

\setlength{\textheight}{9in} \setlength{\topmargin}{-.5in}
\setlength{\textwidth}{6.5in} \setlength{\oddsidemargin}{0in}
\setlength{\evensidemargin}{0in}

\title{Syllabus - COMP401 - Senior Project - Research}
\author{}
\date{Spring 2016}

\begin{document}
\maketitle

\section{Logistics}
\begin{itemize}
\item \textbf{Where: } Center for Science and Business, Room 303
\item \textbf{When: } Th, 12-12:50pm
\item \textbf{Instructors: }
\newline
\newline
\begin{tabular}{l}
\begin{tabular}{l}
\textbf{James \textit{Logan} Mayfield} \\
\textit{Office: }Center for Science and Business, Room 344 \\
\textit{Phone: } 309-457-2200 \\
\textit{Email: } lmayfield at MONMOUTHCOLLEGE dot EDU \\
\textit{Office Hours: } See posting by Office Door or by Appointment.
\end{tabular}
\\
\newline \\
\begin{tabular}{l}
\textbf{Marta Tucker} \\
\textit{Office: } Center for Science and Business, Room 343  \\
\textit{Phone: } 309-457-2354 \\
\textit{Email: } marta at MONMOUTHCOLLEGE dot EDU \\
\textit{Office Hours: }  See posting by Office Door or by Appointment.
\end{tabular} 
\end{tabular}
\item \textbf{Website: } \url{http://jlmayfield.github.io/MC-COMP401-402/}
\item \textbf{Credits: } $\dfrac{1}{2}$ course credit
\end{itemize}


\section{Course Content and Goals}

The senior project is the culminating experience of a student's major in Computer Science and draws upon everything the student has learned over the course of their studies. The project itself is a means to an end and not the ultimate goal of the capstone experience. Sufficiently interesting and complex projects rely on an abundance of existing research and fundamental principles of computing.  The project is, from this perspective, a concrete instantiation of these ideas and principles. By carrying out the project and presenting their work to technical and non-technical audiences students can demonstrate their understanding and mastery of some core element of the computing sciences as it appears in the real world context of their project. 

Students are ultimately working towards pinpointing general, abstract, or theoretical concepts that support their project and clearly articulating how their work is a specific instance of these concepts. An important part of this process is the identification of seminal scholarly work that addresses the concept and its applications. Students too often attempt to reinvent the wheel in the course of their capstone work. Occasionally they're unaware that the wheel already exists. The emphasis on fundamentals is about identifying the wheels, the documents that lay out the general principles of those wheels, and focusing on applying those principles to the specific needs of the project. In doing so the student will better understand where their work sits in the broad spectrum of computing and can present it as such. 

COMP 401 is focused on developing a detailed proposal for the senior project where in the project's place in the wide arena of computing is clear and a workable plan for completing the project in COMP402 is established. Students will take the semester to research topics surrounding their project, identify the wider context of computing in which their work fits, and prepare themselves to immediately begin implementing their proposal the following semester in COMP 402.  Throughout the semester, students will make regular checkpoint presentations demonstrating their progress.  At the end of the semester, students well present their proposed project to a general audience.

\section{Attendance and Expectations}

Students in this course are expected to be respectful of their peers and the instructor. As this course is comprised entirely of student presentations, it is crucial that all students are always present and always on time.  Failure to arrive on time and be a productive member of the course will have a detrimental effect on the final grade and leaves a bad impression with faculty that are likely targets for job and graduate school recommendations.

\section{Course Deliverables}

The following elements of COMP401 contribute to the overall capstone grade:
\begin{itemize}
\item Checkpoint presentations
\item A technical presentation
\item A written proposal
\item A oral presentation of the proposal
\end{itemize}


\subsection{Checkpoint Presentations}

Students can expect to give a 5 to 7 minutes checkpoint presentation on a nearly weekly basis. Checkpoints correspond to sections of the written proposal and students can expect to have at least one checkpoint per section. Refer to the proposal documents for more details about each checkpoint topic. 

At each checkpoint presentation, students are expected to demonstrate progress towards understanding the week's topic as it relates to your project.  These presentations are meant to be an opportunity to get feedback from peers and the instructors.  A students understanding of the topic need not be complete and they are expected to evolve as the semester progresses. 

\subsection{Technical Presentation and Demonstration}

After midterm, students will give a 10 to 20 minute technical presentation and demonstration. The topic should directly address a foundational piece of computer science that relates to their project and do so through the direct application of tools and techniques students intend to use in COMP402. If the project uses a specific programming language or platform, then students might present a program written in that language or for that platform. If the project is written research of a theoretical nature, then students might present a proof utilizing key proof techniques from that research domain. If the research is more experimental, then the students should present an experiment of the type they expect to perform in their project. The key factor is that the subject matter is a relevant foundational principle in computing and the means by which you present that topic are through the tools and techniques you need to use to complete your project next semester. 


\subsection{Written and Oral Proposal}

The details of the proposal are given in a separate document which will be handed out with the syllabus and can be found online on the course website.

\section{Grades}

Students will receive an IP for this course at the completion of the semester. When COMP 402 is completed and the capstone experience is done, a final grade will be determined and applied to both COMP401 and COMP402. Students will receive regular feedback about their standing and are always welcome to discuss their current grades with one or both of the instructors. Grades will be determined based on the following items:
\begin{itemize}
\item COMP401 checkpoints
\item COMP401 Technical Presentation
\item CMP401 Written Proposal
\item COMP401 Proposal Presentation
\item COMP402 Checkpoint Presentations 
\item COMP 402 Research Poster and Scholar's Day Participation 
\item COMP 402 Final Presentation 
\item A Completed Project
\item A Project Bibliography
\end{itemize}

More abstractly, what all of the above elements reflect is a student's: 
\begin{itemize}
\item effective use of technical and problem solving skills befitting a major in Computer Science
\item professionalism
\item ability to make informed, mature decisions as they relate to a larger-scale project
\item understanding and appreciation of the computing disciplines
\end{itemize}


\section{Schedule}

Checkpoint presentations will occur on roughly a weekly basis. Unless otherwise specified, the topics listed below are the topics of checkpoint presentations. These topics correspond to some or all of sections of the final proposal. \textit{This calendar is subject to change based on the circumstances of the course.}

\begin{center}
\begin{tabular}{|c|c|r|}
\hline 
Week & Dates & Assignments \\
\hline
1 & 1/11 - 1/15 &  First Meeting\\
\hline
2 & 1/18 - 1/22 & Rough Project Ideas \\
\hline
3 & 1/25 - 1/29 & Background: Project and Problem  \\
\hline
4 & 2/1 - 2/5 & Background: Technical and Social   \\
\hline
5 & 2/8 - 2/12 & Foundations 1 \\
\hline
6 & 2/15 - 2/19 & Foundations 2  \\
\hline
7 & 2/22 - 2/26 &  Tech-Talk Proposal  \\
\hline
8 & 2/29 - 3/4 &   \\
\hline 
SPRING BREAK & 3/7 - 3/11 &  \\
\hline
9 & 3/14 - 3/18 &  Tech-Talk \\
\hline
10 EASTER BREAK (Fr)& 3/21 - 3/24 & \\
\hline
11 EASTER BREAK (Mo)& 3/29 - 4/1 & Description \& Analysis: Feature and Specifications\\
\hline
12 & 4/4 - 4/8 & Description \& Analysis: System Analysis  \\
\hline
13 & 4/11 - 4/15 &  COMP402 Plan \& Timeline  \\
\hline
14 & 4/18 - 4/22 &  Loose Ends (if needed) \\
\hline
15 FOUNDER'S DAY (Tu) & 4/25 - 4/29 & Written Proposal Due. Presentation\\ 
\hline
16 & 5/2 - 5/4 & \\
\hline
Final's Week & 5/6 (3:00pm) &  \\ 
\hline
\end{tabular}
\end{center}

\subsection{Course Engagement Expectations}

The weekly workload for this course will vary by student but on average should be about 5-7 hours per week.  While regular class meetings are scheduled for two hours a week, it is unlikely that we'll use all of that time each week.  We therefore expect students to dedicate at least 4-6 hours a week towards the development of their project proposals.  This time can include research, preliminary coding, writing, meeting with professors, and so forth.



\end{document}

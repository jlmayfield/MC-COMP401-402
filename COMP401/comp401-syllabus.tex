\documentclass[10pt]{article}
\usepackage{amsmath}
\usepackage{setspace}
%\usepackage{multirow}

\setlength{\textheight}{9in} \setlength{\topmargin}{-.5in}
\setlength{\textwidth}{6.5in} \setlength{\oddsidemargin}{0in}
\setlength{\evensidemargin}{0in}

\title{Syllabus - COMP401 - Senior Project - Research}
\author{}
\date{Spring 2015}

\begin{document}
\maketitle

\section{Logistics}
\begin{itemize}
\item \textbf{Where: } Center for Science and Business, Room 309
\item \textbf{When: } Th, 11-11:50
\item \textbf{Instructors: }
\newline
\newline
\begin{tabular}{l}
\begin{tabular}{l}
\textbf{James \textit{Logan} Mayfield} \\
\textit{Office: }Center for Science and Business, Room 344 \\
\textit{Phone: } 309-457-2200 \\
\textit{Email: } lmayfield at MONMOUTHCOLLEGE dot EDU \\
\textit{Office Hours: } By Appointment.
\end{tabular}
\\
\newline \\
\begin{tabular}{l}
\textbf{Marta Tucker} \\
\textit{Office: } Center for Science and Business, Room 343  \\
\textit{Phone: } 309-457-2354 \\
\textit{Email: } marta at MONMOUTHCOLLEGE dot EDU \\
\textit{Office Hours: }  See posting by Office Door or by Appointment.
\end{tabular} 
\\
\newline \\
\begin{tabular}{l}
\textbf{Fabio Guerinoni} \\
\textit{Office: } Center for Science and Business, Room 342  \\
\textit{Phone: } 309-457-2247 \\
\textit{Email: } fguerinoni at MONMOUTHCOLLEGE dot EDU \\
\textit{Office Hours: }  See posting by Office Door or by Appointment.
\end{tabular} 
\end{tabular}

\item \textbf{Credits: } $\dfrac{1}{2}$ course credit
\end{itemize}


\section{Course Content and Goals}

The senior project is designed to be the culminating experience of a Computer Science major program.  It calls upon students to draw on everything they have learned over the course of their studies. Students work in either small groups individually to plan and carry out a major research or development project.  

COMP 401 is focused on developing a detailed proposal for the senior project. Students will take the semester to research topics surrounding their project and prepare themselves to immediately begin implementing their proposal the following spring.  Through out the semester, students will make regular checkpoint presentations demonstrating their progress.  At the end of the semester, students well present their proposed project to a general audience as well as prepare a poster highlighting the work they plan to carry out in the spring. 

The overall goals are that:
\begin{itemize}
\item Students will plan a major computer science research or development project
\item Students will prepare to undertake their project when they return in the spring by gaining working knowledge of the technology and previous work upon which their project relies
\item Students will begin to form a understanding of the community of researchers and developers in which their work places them
\item Students will begin to form a understanding of the impact that their current and future work will have on society at large 
\end{itemize}

\section{Attendance and Expectations}

Students in this course are expected to be respectful of their peers and the instructor. As this course is comprised entirely of student presentations, it is crucial that you are always in class and always on time.  Come prepared to listen, discuss, and present.  Failure to arrive on time and be a productive member of the course will have a detrimental effect on your grade.  Not only that, but it leaves a certain impression on the faculty from whom you'll soon want letters of recommendation for jobs.  

\section{Checkpoint Presentations}

Students can expect to give a checkpoint presentation on a nearly weekly basis. In general, each checkpoint corresponds to a section of the final proposal. The schedule for checkpoints can be found below. You should refer to the proposal documents for more details about each checkpoint topic. 

At each checkpoint presentation, you are expected to demonstrate progress towards understanding the week's topic as it relates to your project.  These presentations are meant to be an opportunity to get feedback from your peers and the instructors.  You understanding of the topic need not be complete and we expect you'll continue to flesh out details as the semester progresses.  Presentations should be prepared ahead of time and should not take more than five to ten minutes. 

\section{Grades}

At the completion of this course, the grade for both COMP401 and COMP402 is determined. Students will typically receive the same grade in both courses to reflect the work throughout the capstone project and not in one individual phase of the project. Grades will be determined based on the following items:
\begin{itemize}
\item Appropriateness of project difficulty (evaluated during COMP401) 
\item COMP401 checkpoints
\item COMP401 Technical Presentation
\item COMP401 Proposal Poster
\item CMP401 Written Proposal
\item COMP401 Proposal Presentation
\item COMP402 Checkpoint Presentations 
\item COMP 402 Research Poster and Scholar's Day Participation 
\item COMP 402 Final Presentation 
\item A Completed Project and required components
\end{itemize}

More abstractly, what all of the above elements should reflect is a student's: 
\begin{itemize}
\item effective use of technical and problem solving skills befitting a major in Computer Science
\item professionalism
\item ability to make informed, mature decisions as they relate to a larger-scale project
\item understanding and appreciation of the computing disciplines
\end{itemize}



\section{Schedule}

Checkpoint presentations will occur on roughly a weekly basis. \textit{This calendar is subject to change based on the circumstances of the course.}
\begin{center}
\begin{tabular}{|c|c|r|}
\hline 
Week & Dates & Assignments \\
\hline
1 & 1/12 - 1/16 & Initial Meeting \\
\hline
2 & 1/19 - 1/23 & Initial Concepts and Ideas\\
\hline
3 & 1/26 - 1/30 & Initial Project Proposal   \\
\hline
4 & 2/2 - 2/6 &  Background and History \\
\hline
5 & 2/9 - 2/13 &  Community \& Stakeholders \\
\hline
6 & 2/16 - 2/20 & Tools \& Bibliography  \\
\hline
7 & 2/23 - 2/27 &  Features \& Key Research \\
\hline
8 & 3/2 - 3/6 &   \\
\hline 
SPRING BREAK & 3/9 - 3/13 &  \\
\hline
9 & 3/16 - 3/20 & Pre-Tech Talk. \\
\hline
10 & 3/23 - 3/27 & Technical Talk \\
\hline
11 & 3/30 - 4/3 & Preliminary Design. EASTER BREAK (Friday).\\
\hline
12 & 4/6 - 4/10 & EASTER BREAK (Monday). Final Design \& Plan \\
\hline
13 & 4/13 - 4/17 &   \\
\hline
14 & 4/20 - 4/24 &  Pre-Proposal Wrap-Up \\
\hline
15 & 4/27 - 5/1 & Proposal Presentation. \\ 
\hline
16 & 5/4 - 5/6 & \\
\hline
Final's Week &  &  \\ 
\hline
\end{tabular}
\end{center}


\subsection{Course Engagement Expectations}

The weekly workload for this course will vary by student but on average should be about 5-7 hours per week.  While regular class meetings are scheduled for two hours a week, it is unlikely that we'll use all of that time each week.  We therefore expect students to dedicate at least 4-6 hours a week towards the development of their project proposals.  This time can include research, preliminary coding, writing, meeting with professors, and so forth.



\end{document}

\documentclass[10pt]{article}
\usepackage{amsmath}
\usepackage{setspace}
\usepackage{hyperref}
\usepackage{booktabs}

\setlength{\textheight}{9in} \setlength{\topmargin}{-.5in}
\setlength{\textwidth}{6.5in} \setlength{\oddsidemargin}{0in}
\setlength{\evensidemargin}{0in}

\title{Syllabus \\ COMP401 --- Senior Project Research}
\author{}
\date{Fall 2018}

\begin{document}
\maketitle

\section{Logistics}
\begin{itemize}
\item \textbf{Where: } Center for Science and Business, Room 303
\item \textbf{When: } Th, 1--1:50pm
\item \textbf{Instructors: }
\newline
\begin{tabular}{l}
\begin{tabular}{l}
\textbf{James \textit{Logan} Mayfield} \\
\textit{Office: }Center for Science and Business, Room 344 \\
\textit{Phone: } 309-457-2200 \\ % chktex 8
\textit{Email: } lmayfield at MONMOUTHCOLLEGE dot EDU \\
\textit{Website: } \url{http://jlmayfield.github.io} \\
\end{tabular}
\\
\newline \\
\begin{tabular}{l}
\textbf{Robert Utterback} \\
\textit{Office: } Center for Science and Business, Room 342  \\
\textit{Phone: }  309-457-2202 \\ % chktex 8
\textit{Email: } rutterback at MONMOUTHCOLLEGE dot EDU \\
\end{tabular}
\end{tabular}
\item \textbf{Course Website: } \url{http://jlmayfield.github.io/teaching/COMP401-402/} % chktex 8
\item \textbf{Credits: } \(\dfrac{1}{2}\) course credit
\end{itemize}


\section{Course Content and Goals}

The senior project is the culminating experience of a student's major in Computer Science and draws upon everything the student has learned over the course of their studies. The project itself is a means to an end and not the ultimate goal of the capstone experience. Sufficiently interesting and complex projects rely on an abundance of existing research and fundamental principles of computing.  The project is, from this perspective, a concrete instantiation of these ideas and principles. By carrying out the project and presenting their work to technical and non-technical audiences students can demonstrate their understanding and mastery of some core element of the computing sciences as it appears in the real world context of their project.

Students are ultimately working towards pinpointing general, abstract, or theoretical concepts that support their project and clearly articulating how their work is a specific instance of these concepts. An important part of this process is the identification of seminal scholarly work that addresses the concept and its applications. Students too often attempt to reinvent the wheel in the course of their capstone work. Occasionally they're unaware that the wheel already exists. The emphasis on fundamentals is about identifying the wheels, the documents that lay out the general principles of those wheels, and focusing on applying those principles to the specific needs of the project. In doing so the student will better understand where their work sits in the broad spectrum of computing and can present it as such.

COMP 401 is focused on developing a detailed proposal for the senior project where in the project's place in the wide arena of computing is clear and a workable plan for completing the project in COMP402 is established. Students will take the semester to research topics surrounding their project, identify the wider context of computing in which their work fits, and prepare themselves to immediately begin implementing their proposal the following semester in COMP 402.  Throughout the semester, students will make regular checkpoint presentations demonstrating their progress.  At the end of the semester, students well present their proposed project to a general audience.

\section{Attendance and Expectations}

Students in this course are expected to be respectful of their peers and the instructor. As this course is comprised entirely of student presentations, it is crucial that all students are always present and always on time.  Failure to arrive on time and be a productive member of the course will have a detrimental effect on the final grade and leaves a bad impression with faculty that are likely targets for job and graduate school recommendations.

\section{Course Deliverables}

The following elements of COMP401 contribute to the overall capstone grade:
\begin{itemize}
\item Checkpoint presentations
\item A technical presentation
\item A written proposal
\item A oral presentation of the proposal
\end{itemize}


\subsection{Checkpoint Presentations}

Students can expect to give a 5 to 7 minutes checkpoint presentation on a nearly weekly basis. Checkpoints correspond to sections of the written proposal and students can expect to have at least one checkpoint per section. Refer to the proposal documents for more details about each checkpoint topic.

At each checkpoint presentation, students are expected to demonstrate progress towards understanding the week's topic as it relates to your project.  These presentations are meant to be an opportunity to get feedback from peers and the instructors.  A students understanding of the topic need not be complete and they are expected to evolve as the semester progresses.

\subsection{Technical Presentation}

After midterm, students will give a 10 to 20 minute technical presentation.  The goal of this presentation is to present the technical details of the concrete instantiation of a foundational piece of computer science in the context of the capstone project. If the project uses a specific programming language or platform, then students might present a program written in that language or for that platform. If the project is written research of a theoretical nature, then students might present a proof utilizing key proof techniques from that research domain. If the research is more experimental, then the students should present an experiment of the type they expect to perform in their project. The key factor is that \textit{the subject matter is a relevant foundational principle in computing and the means by which you present that topic are through the tools and techniques you need to use to complete your project next semester}.


\subsection{Written and Oral Proposal}

The details of the proposal are given in a separate document which will be handed out with the syllabus and can be found online on the course website.

\section{Grades}

Students will receive an IP for this course at the completion of the semester. When COMP 402 is completed and the capstone experience is done, a final grade will be determined and applied to both COMP401 and COMP402. Students will receive regular feedback about their standing and are always welcome to discuss their current grades with one or both of the instructors. Grades will be determined based on the following items:
\begin{itemize}
\item COMP401 checkpoints
\item COMP401 Technical Presentation
\item CMP401 Written Proposal
\item COMP401 Proposal Presentation
\item COMP402 Checkpoint Presentations
\item COMP 402 Research Poster and Scholar's Day Participation
\item COMP 402 Final Presentation
\item A Completed Project
\item A Project Bibliography
\end{itemize}

More abstractly, what all of the above elements reflect is a student's:
\begin{itemize}
\item effective use of technical and problem solving skills befitting a major in Computer Science
\item professionalism
\item ability to make informed, mature decisions as they relate to a larger-scale project
\item understanding and appreciation of the computing disciplines
\end{itemize}

The following examples provide an idea of what we are looking for in
the above items. \textbf{These are rough guidelines only!} Assessment
can vary considerably depending on the project.

\begin{itemize}
\item \textbf{A/A-/B+ range}. Steady progress is made (more or less)
  from start to finish. The student takes ownership of the project and
  approaches it the right way, with an occasional mishap. The real key
  is that \textbf{progress can be explained and justified by the
    student}.
\item \textbf{B range}. Attempts are made to explain/justify work, but
  explanations often lack a certain level of understanding. Students
  with projects in this range are clearly saying what they think they
  need to say, sometimes reciting technical information taken verbatim
  from various sources.
\item \textbf{B-/C+ range}. Little to no real planning and research in
  401. An idea is proposed and the student talks a lot about what they
  want to do, but they never really engage in the problems they'll
  encounter when developing the project. But the student regroups in
  402, working the right way and providing some level of understanding
  in their work.
\item \textbf{C range}. Little to no real planning and research in 401
  that carries through to 402. The student is unable to explain and
  justify their work beyond showing that something happens. Something
  works, but there are lots of bugs or missing features, sometimes
  including large components.
\item \textbf{C- range or worse} Little appreciable progress
  throughout 401 and 402. The student is unable to explain what they
  did, nor what is really going on in the code. The code works
  sometimes but not most times. The student shows clear signs of
  stringing together code or material from other sources with no
  understanding.
\end{itemize}

\section{Schedule}

%This course is run concurrently with COMP 402. Both schedules are given below.

 Checkpoint presentations will occur on roughly a weekly basis. Unless otherwise specified, the topics listed below are the topics of checkpoint presentations. These topics correspond to some or all of sections of the final proposal. \textit{This calendar is subject to change based on the circumstances of the course.}

\begin{center}
\begin{tabular}{lll}
\underline{Week} & \underline{Dates} &  \underline{COMP401} \\
1 & 8/22 --- 8/24 & Initial Meeting   \\
2 & 8/27 --- 8/31 & Project Ideas   \\
3 & 9/3 --- 9/7 & Background: Project \& Problems   \\
4 & 9/10 --- 9/14 & Background: Technical\& Social    \\
5 & 9/17 --- 9/21 & Foundations  \\
6 & 9/24 --- 9/28 & Foundations   \\
7 & 10/2 --- 10/6 & Tech-Talk Proposal  \\
8 & 10/8 --- 10/10 & FALL BREAK (WThF)  \\
9 & 10/16 --- 10/19 & Tech-Talk Check-In   \\
10 & 10/22 --- 10/26 & Tech-Talk  \\
11 & 10/29 --- 11/2 &  Features \& Specifications   \\
12 & 11/5 --- 11/9 & Plan \& Timeline  \\
13 & 11/12 --- 11/16 &   \\
14 & 11/19 --- 11/20  & THANKSGIVING (WThF) \\
15 & 11/26 --- 12/5 & Proposal Due. Presentation.  \\
16 & 12/3 --- 12/5 &  \\
Final's Week &  &    \\
\end{tabular}
\end{center}


\subsection{Course Engagement Expectations}

The weekly workload for this course will vary by student but on average should be about 5--7 hours per week.  While regular class meetings are scheduled for two hours a week, it is unlikely that we'll use all of that time each week.  We therefore expect students to dedicate at least 4--6 hours a week towards the development of their project proposals.  This time can include research, preliminary coding, writing, meeting with professors, and so forth.



\end{document}

\documentclass[10pt]{article}
\usepackage{amsmath}
\usepackage{setspace}
\usepackage{hyperref}

\setlength{\textheight}{9in} \setlength{\topmargin}{-.5in}
\setlength{\textwidth}{6.5in} \setlength{\oddsidemargin}{0in}
\setlength{\evensidemargin}{0in}

\title{Syllabus - COMP401 - Senior Project - Research}
\author{}
\date{Fall 2015}

\begin{document}
\maketitle

\section{Logistics}
\begin{itemize}
\item \textbf{Where: } Center for Science and Business, Room 303
\item \textbf{When: } Th, 11-11:50
\item \textbf{Instructors: }
\newline
\newline
\begin{tabular}{l}
\begin{tabular}{l}
\textbf{James \textit{Logan} Mayfield} \\
\textit{Office: }Center for Science and Business, Room 344 \\
\textit{Phone: } 309-457-2200 \\
\textit{Email: } lmayfield at MONMOUTHCOLLEGE dot EDU \\
\textit{Office Hours: } See posting by Office Door or by Appointment.
\end{tabular}
\\
\newline \\
\begin{tabular}{l}
\textbf{Marta Tucker} \\
\textit{Office: } Center for Science and Business, Room 343  \\
\textit{Phone: } 309-457-2354 \\
\textit{Email: } marta at MONMOUTHCOLLEGE dot EDU \\
\textit{Office Hours: }  See posting by Office Door or by Appointment.
\end{tabular} 
\end{tabular}
\item \textbf{Website: } \url{http://jlmayfield.github.io/MC-COMP401-402/}
\item \textbf{Credits: } $\dfrac{1}{2}$ course credit
\end{itemize}


\section{Course Content and Goals}

The senior project is designed to be the culminating experience of a Computer Science major program.  It calls upon students to draw on everything they have learned over the course of their studies. Students work in either small groups individually to plan and carry out a major research or development project.  

The project itself is a means to an end and not the ultimate goal of the capstone experience. Sufficiently interesting and complex projects rely on an abundance of existing research and fundamental principals of computing.  What we want is for you to identify and understand this context, learn to work in this context by way of completing your project, and ultimately to communicate these ideas to technical and non-technical people by using the specifics of your project as the concrete instantiation of this context. 
 
COMP 401 is focused on developing a detailed proposal for the senior project. Students will take the semester to research topics surrounding their project, identify the wider context of computing in which their work fits, and prepare themselves to immediately begin implementing their proposal the following semester in COMP 402.  Through out the semester, students will make regular checkpoint presentations demonstrating their progress.  At the end of the semester, students well present their proposed project to a general audience as well as prepare a small poster highlighting the work they plan to carry out in the spring. 

The overall goals are that:
\begin{itemize}
\item Students will plan a major computer science research or development project
\item Students will prepare to undertake their project when they return in the spring by gaining working knowledge of the technology and previous work upon which their project relies
\item Students will begin to form a understanding of the community of researchers and developers in which their work places them
\item Students will begin to form a understanding of the impact that their current and future work may  have on society at large 
\end{itemize}

Do not think of this work as yet another school project. Instead, think of it as a real foray into the professional realm of computing. 

\section{Attendance and Expectations}

Students in this course are expected to be respectful of their peers and the instructor. As this course is comprised entirely of student presentations, it is crucial that you are always in class and always on time.  Come prepared to listen, discuss, and present.  Failure to arrive on time and be a productive member of the course will have a detrimental effect on your grade.  Not only that, but it leaves a certain impression on the faculty from whom you'll soon want letters of recommendation for jobs.  

\section{Checkpoint Presentations}

Students can expect to give a checkpoint presentation on a nearly weekly basis. In general, each checkpoint corresponds to a section of the final proposal. The schedule for checkpoints can be found below. You should refer to the proposal documents for more details about each checkpoint topic. 

At each checkpoint presentation, you are expected to demonstrate progress towards understanding the week's topic as it relates to your project.  These presentations are meant to be an opportunity to get feedback from your peers and the instructors.  You understanding of the topic need not be complete and we expect you'll continue to flesh out details as the semester progresses.  Presentations should be prepared ahead of time and should not take more than five to ten minutes. 

\section{Grades}

You will receive an IP for this course at the completion of the semester. When COMP 402 is completed and the capstone experience is done, your final grade will be determined and applied to both COMP401 and COMP402. Students will receive regular feedback about the standing and are always welcome to discuss their current grades with one or both of the instructors. Grades will be determined based on the following items:
\begin{itemize}
\item Appropriateness of project difficulty (evaluated during COMP401) 
\item COMP401 checkpoints
\item COMP401 Technical Presentation
\item COMP401 Proposal Poster
\item CMP401 Written Proposal
\item COMP401 Proposal Presentation
\item COMP402 Checkpoint Presentations 
\item COMP 402 Research Poster and Scholar's Day Participation 
\item COMP 402 Final Presentation 
\item A Completed Project and required components
\end{itemize}

More abstractly, what all of the above elements should reflect is a student's: 
\begin{itemize}
\item effective use of technical and problem solving skills befitting a major in Computer Science
\item professionalism
\item ability to make informed, mature decisions as they relate to a larger-scale project
\item understanding and appreciation of the computing disciplines
\end{itemize}


\section{Schedule}

Checkpoint presentations will occur on roughly a weekly basis. Unless otherwise specified, the topics listed below are the topics of checkpoint presentations. These topics correspond to some or all of sections of the final proposal. \textit{This calendar is subject to change based on the circumstances of the course.}

\begin{center}
\begin{tabular}{|c|c|r|}
\hline 
Week & Dates & Assignments \\
\hline
1 & 8/25 - 8/28 &   \\
\hline 
2 & 8/31 - 9/4 &   Brainstorming \\
\hline
3 & 9/7 - 9/11 &   Background: Project and Problem \\
\hline
4 & 9/14 - 9/18 &  Background: Technical Context \\
\hline
5 & 9/21 - 9/25 &  Background: Social Context \\
\hline
6 & 9/28 - 10/2 &  Foundations \\
\hline
7 & 10/5 - 10/9  & Tech-talk Proposals\\
\hline 
8 & 10/12 - 10/15 &  FALL BREAK (F) \\
\hline
9 & 10/21 - 10/23 & FALL BREAK (M,Tu) Technical Talk/Demo \\
\hline
10 & 10/26 - 10/30 &  \\
\hline
11 & 11/2 - 11/6 & Project Description and Analysis \\
\hline
12 & 11/9 - 11/13 & Project Description and Analysis \\
\hline
13 & 11/16 - 11/20 & 402 Time Line \\ 
\hline
14 & 11/23 - 11/24 &  Written Proposals Due. THANKSGIVING BREAK (W-F).   \\
\hline
15 & 11/30 - 12/4 & Proposal Presentations\\ 
\hline
16 & 12/7 - 12/9 &   Reading Day (Th). \\
\hline
Final's Week & 12/12  \\ 
\hline
\end{tabular}
\end{center}

\subsection{Course Engagement Expectations}

The weekly workload for this course will vary by student but on average should be about 5-7 hours per week.  While regular class meetings are scheduled for two hours a week, it is unlikely that we'll use all of that time each week.  We therefore expect students to dedicate at least 4-6 hours a week towards the development of their project proposals.  This time can include research, preliminary coding, writing, meeting with professors, and so forth.



\end{document}

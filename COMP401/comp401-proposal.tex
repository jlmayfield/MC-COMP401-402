\documentclass[10pt]{article}
\usepackage{amsmath}
\usepackage{setspace}
\usepackage{hyperref}


\setlength{\textheight}{9in} \setlength{\topmargin}{-.5in}
\setlength{\textwidth}{6.5in} \setlength{\oddsidemargin}{0in}
\setlength{\evensidemargin}{0in}

\title{COMP 401 - Project Proposal}
\author{  }
\date{Fall 2015}


\begin{document}
\maketitle

COMP401 culminates with the formal proposal of your senior capstone project.  The proposal consists of three pieces
\begin{enumerate}
\item Written Project Proposal
\item Presentation
\item Mini-Poster
\end{enumerate}
Your presentation is a summary of the written proposal given in a public forum while the poster serves as a condensed snap shot of the project you'll be implementing in COMP402.  Neither should introduce information not found in the written proposal.

\section{Written Proposal}

Your written proposal should contain the sections listed below.  Each section has an estimated page length next to it.  Page length estimates assume \textit{single spaced, 11 pt. font}.  Please note that these are estimates and the most important thing is that you're detailed enough for your particular proposal.    
\begin{enumerate}
\item Abstract [1-2]
\item Background/Introduction [5-8]
\item Project Description [5-15]
\item Foundations [5-10] 
\item Implementation Plan and Time Line [2-5]
\item Conclusion [1-2]
\item Bibliography \& Works Cited [1-2]
\end{enumerate}

\subsection{Abstract}

The abstract should be roughly one to two pages and should give a high level summary of the document.  One should be able to read your abstract and determine what your project is and why it's interesting/important.  The goal is to get the reader interested in the work and believing that you have a firm, actionable plan for carrying it out next semester. In doing so, you should be summarizing the fleshed out document that follows. 

\subsection{Background/Introduction}

This section introduces the project and the problems and ideas that led to its conception.  In doing so you begin to set your project in a wider socio-technical context. You should address:
\begin{itemize}
\item The problem(s) (social, technical, or otherwise) addressed by your project.
\item How, specifically, your project addresses/solves the problem.
\item What work is related to yours and what niche filled by your project. Meaning, if similar work is out there, then how is yours different/better, or if your work is totally unique, then to what do you attribute this uniqueness. This discussion should focus on current work and not on deeply foundational work, which you'll cover later.   
\end{itemize}
This section should focus on the high-level, big-picture details of your project. 

\subsection{Project Description and Analysis}

In this section you need to paint a very clear and detailed picture of what it is you'll be doing. Your discussion and analysis is likely to include:
\begin{itemize}
\item Concrete use cases for your work
\item A clear, and explicit discussion of the features of your work
\item Analysis of users and stakeholders for your work
\item A look at the ethics of your work
\item A high to mid-level system design and architecture 
\end{itemize}
The key here is to clearly lay out the \textit{what} of your project, not the \textit{how}.
After reading this section it should be \textit{crystal clear} what your work will and will not contain. This section should not delve deeply into the exact means by which you'll complete the project as proposed; that comes later. Do not shy away from pictures, diagrams, charts, tables, and other means of visualization. 

\subsection{Foundations}

In this section you'll lay out the fundamental research and computing principals that support your work. You should identify seminal research and lay out the path by which those results travel to get to your work. Any project of sufficient complexity should touch upon several veins of research. As such, this section should lay out multiple foundations for your project. You should survey these foundations, discuss the key papers, and clearly layout the relationship between the foundations and the concrete details of your project. 

\subsection{Implementation Plan and Time Line}

It is in this section where you should clearly lay down how you will complete your work. This should include a \textit{detailed} time line for your work in COMP402. Your discussion is likely to include:
\begin{itemize} 
\item Tools (Software, Languages, Libraries, etc.) to be used
\item Resources and Documentation to be used
\item Testing/Editing Plans including 
\item Priorities and plans for scaling back if needed
\end{itemize}
By this point in the proposal readers should be convinced you've thought about what you're going to do with sufficient detail. Now you must convince them that you have a realistic and actionable plan for COMP 402. The plan you lay out here provides a path forward for COMP 402. It's very possible that in the context of next semester you'll need or want to deviate from the path. That's OK. What is not OK is that you feel your way through the dark in COMP 402. 


\subsection{Conclusion}

The final section of your proposal should be both a summary of the proposal and a place for you to reflect on the work ahead of you.  Be certain to write about:
\begin{itemize}
\item The importance/significance of your work
\item The technical challenges that you foresee in your work
\item The personal challenges that you foresee in carrying out your work
\end{itemize}
Make it clear to the reader that this work is worth doing,that you are fully aware of the technical and personal challenges that await you, and that you're ready to face them head on. 


\subsection{Bibliography \& Works Citied}

Any information that is not your own and is used in the writing of your proposal should be properly cited.  If you refer to technical details about software/hardware that you yourself did not create, then you should cite a reliable source for this information.  The key components of your bibliography are: 
\begin{itemize}
\item Resources and documentation needed for working with new tools
\item Foundational research
\end{itemize}

\section{Presentation}

You'll be giving a \textit{twenty to thirty minute} presentation in which you propose your work. It is essentially the oral form of your paper. The goal of the proposal is to be certain that you have a clear path laid out for success.  The presentation is the time when you convince us that what you wrote is more than just words on paper. There will be time for questions after your presentation. Be prepared to explain and re-explain plans and details of your project.  

Here's a rough outline of what your presentation should entail with time estimates, in minutes, for each item.  The information for each section should mirror/summarize the relevant section from the written proposal. 
\begin{itemize}
\item Background/Introduction [2-5]
\item Project Description [5-7]
\item Foundations [5-7]
\item Plan for 402 [5-7]
\item Conclusion [2-3]
\end{itemize}

You must have Power Point slides or some comparable visual aides to accompany the presentation.

\section{Poster}

Your poster will act as a snapshot of the written proposal but without a majority of the technical details.  It touches on information from the following sections of your proposal.
\begin{itemize}
\item Background/Introduction
\item Project Description
\item Foundations
\end{itemize}
Posters need to be 11x17 inches and will be hung in Math/CS wing of CSB while you are working on the project. They are intended to get people excited about your work.


\end{document}
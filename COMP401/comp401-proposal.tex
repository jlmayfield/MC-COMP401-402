\documentclass[nobib]{tufte-handout}
\usepackage{amsmath}
\usepackage{setspace}
\usepackage{hyperref}
\usepackage{booktabs}


%\setlength{\textheight}{9in} \setlength{\topmargin}{-.5in}
%\setlength{\textwidth}{6.5in} \setlength{\oddsidemargin}{0in}
%\setlength{\evensidemargin}{0in}

\title{COMP 401 --- Project Proposal}
\author{  }
\date{Fall 2017}


\begin{document}
\maketitle

COMP401 culminates with the formal proposal of your senior capstone project.  The proposal is given in two forms:
\begin{enumerate}
\item Written Project Proposal
\item Proposal Presentation
\end{enumerate}
The presentation is effectively a summary of the written proposal given in a public forum.

\section{Written Proposal}

The written proposal should contain the sections listed below.  Each section has an estimated page length next to it.  Page length estimates assume \textit{single spaced, 11 pt.\ font}.  Please note that these are estimates and the most important thing is that each section is detailed enough for project at hand.
\begin{center}
\begin{tabular}{ll}
  Section & Length (pages) \\ \toprule
  Abstract & 1--2 \\
  Background/Introduction  & 3--5 \\
  Project Description  & 5--10 \\
  Foundations & 10--15 \\
  Implementation Plan and Time Line & 2--5 \\
  Conclusion & 1--2 \\
  Bibliography \& Works Cited & 2--4 \\ \midrule
   & 24--43
\end{tabular}
\end{center}

\subsection{Abstract}

The abstract should be roughly one to two pages and should give a high level summary of the document.  One should be able to read the abstract and determine what the project is and why it's interesting and important.  The goal is to get the reader interested in the work and believing that the proposer has a firm, actionable plan for carrying it out next semester.

\subsection{Background \& Introduction}

This section introduces the project along with the problems and ideas that led to its conception.  The emphasis here is on the end product.  To whom is this project interesting and important and why? The goal is to establish the wider socio-technical context in which that end product exists, namely:
\begin{itemize}
\item The problem or problems (social, technical, or otherwise) addressed by the project.
\item How, specifically, the project addresses or solves the problem.
\item What work is related or comparable to the project and what niche is filled by the project. Meaning, if similar work is out there, then how is this project different?  If this project is totally unique, then to what is it's uniqueness attributed?
\end{itemize}


\subsection{Project Description and Analysis}


This section lays down the technical details for the project. Where the background section spells out the what and the why of the project, this section begins the process of explaining, at least in the abstract, the how of the project. What exactly needs to go in here will vary from project to project.  Likely candidates include:
\begin{itemize}
\item A clear, and explicit discussion of the features of the project
\item Concrete use cases for the project.
\item A high to mid-level system design and architecture
\item An outline of a paper
\item Experimental plans
\end{itemize}
When all is said and done, the reader should have a very concrete idea of what the project entails after reading this section and feel confident that this project, if completed, would address the issues discussed in the previous section.

\subsection{Foundations}

This section continues establishing the \textit{how} of your project by connecting the high level description of the previous section to foundational research and ideas in computing. Students should:
\begin{itemize}
\item Identify seminal research related to their work
\item Discuss the specific instantiation of that research in the confines of their project
\item Assess the difficulty of a project component with respect to the status of existing research
\end{itemize}
Where the project description connects the real-world problems and issues to concrete components of the project, this section establishes the feasibility of those components by exploring their status as solved or open problems.


\subsection{Implementation Plan and Time Line}

It is in this section students lay down the path they intended to take to get from start to finish. This should include:
\begin{itemize}
\item A \textit{detailed} time line from COMP402 including bi-weekly checkpoints.
\item A discussion of tools (Software, Languages, Libraries, etc.) to be used
\item Testing/Editing/Revising Plans
\item Priorities and plans for scaling back if needed
\end{itemize}
The previous sections should convince the reader that the proposer has explored the project to a depth sufficient for having a realistic sense of what needs to get done and its feasibility. This section then explains how that work can get done over the course of a semester.


\subsection{Conclusion}

The final section of the proposal should be both a summary of the proposal and a place to reflect on the work ahead. The reader needs to be assured that this project is worth doing and that the proposer is capable of doing it.


\subsection{Bibliography \& Works Citied}

All proposals should contain some amount of inline citations as they'll need to refer to previous research and existing tool such as libraries and languages. All of these cited documents and works should appear in the bibliography. In addition to these sources, students should clearly list key resources, documentation, and foundation research they expect to utilize when carrying out their work in COMP402.

\section{Presentation}

Students will give a \textit{20 to 30 minute} presentation of their proposed work. It is essentially the oral form of the written proposal. The presentation outline should mirror the paper's outline. The following are rough estimates of the number of minutes one might expect to spend per section.
\begin{center}
  \begin{tabular}{ll}
    Section & Length (min) \\ \toprule
    Background/Introduction & 2--5 \\
    Project Description & 5--7 \\
    Foundations & 7--10 \\
    Plan for COMP402 & 3--5 \\
    Conclusion & 2--3
  \end{tabular}
\end{center}

There will be time for questions after the presentation, and students should be prepared to explain and re-explain the plans and details of the project.

\end{document}

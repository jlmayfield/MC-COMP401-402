\documentclass[10pt]{article}
\usepackage{amsmath}
\usepackage{setspace}
\usepackage{hyperref}
\usepackage{multirow}

\setlength{\textheight}{9in} \setlength{\topmargin}{-.5in}
\setlength{\textwidth}{6.5in} \setlength{\oddsidemargin}{0in}
\setlength{\evensidemargin}{0in}

\title{Syllabus - COMP402 - Senior Project - Implementation}
\author{James \textit{Logan} Mayfield}
\date{ Spring 2014}

\begin{document}
\maketitle

\section{Logistics}
\begin{itemize}
\item \textbf{Where: } Center for Science \& Business, Room 303.
\item \textbf{When: } Th 11-11:50 am 
\item \textbf{Instructors: }
\newline
\newline
\begin{tabular}{l}
\begin{tabular}{l}
\textbf{James \textit{Logan} Mayfield} \\
\textit{Office: } Center for Science and Business, Room 344. \\
\textit{Phone: } 309-457-2200 \\
\textit{Email: } lmayfield at monmouthcollege dot edu \\
\textit{Office Hours: } By Appointment.
\end{tabular}
\\
\newline \\
\begin{tabular}{l}
\textbf{Marta Tucker} \\
\textit{Office: } Center for Science and Business, Room 345. \\
\textit{Phone: } 309-457-2354 \\
\textit{Email: } marta at monmouthcollege dot edu \\
\textit{Office Hours}: By Appointment
\end{tabular} 
\end{tabular}

\item \textbf{Credits: } $\dfrac{1}{2}$ course credit
\end{itemize}

\section{Course Content and Goals}

The senior project is designed allow students to demonstrate their abilities to apply all that they have learned about Computer Science and thereby act as a culminating experience for their studies in computing at Monmouth College.  Depending on class size, either groups of students or each individual student will be responsible for planning and carrying out a Computer Science related project.  

COMP 402 is focused on implementation of the plans proposed by the student in COMP401.  The class will meet on a semi-regular basis for  brief presentations on the current state of the projects in order to receive feed back from peers and faculty.  At the end of the semester, students will present the final results of their work to the campus at large and at the Science Poster session held each spring on Scholar's Day.


\section{The Project Components}

Students projects generally fall into one of two categories: Software-Based projects or research projects.  Software based projects are generally programming centric and result in working software.  Research projects fall into line with traditional scientific research projects and generally result in a paper. 

\subsection{Software-Based Projects}

Software-based projects must include:
\begin{itemize}
\item A working, publicly available final product
\item Source code with relevant documentation
\item End-User documentation
\item A poster fit for a gathering of other software developers
\end{itemize}
Your project does not need to be open-source but you must submit your source to the course instructors.

Software may be made publicly available in many different ways.  A few options include:
\begin{itemize}
\item Hosting the code and executable as an open-source project on a site such as \url{http://github.com}
\item Hosting final product, possibly closed-source, on software hosting/download site such as an App Store.
\end{itemize}
\textit{Students may ``publish'' their work in other ways, but must get the OK from the instructors before doing so.}

\subsection{Research Projects}

Research projects must include:
\begin{itemize}
\item A complete, published paper 
\item An annotated bibliography and works cited
\item A research-style poster
\end{itemize}

Publication of final papers need not be in a peer-reviewed journal or conference proceeding.  The following are a few examples of ways in which students might meet their publication requirements:
\begin{itemize}
\item Submission to a peer-reviewed journal or conference with no requirement for acceptance
\item Submission to a reputable pre-print archive such as \url{http://www.arxiv.org}
\end{itemize}
\textit{Students may ``publish'' their work in other ways, but must get the OK from the instructors before doing so.}

\section{Schedule}

The class will, more often than not, meet on a bi-weekly basis.
\begin{center}
\begin{tabular}{|c|c|r|}
\hline 
Week & Dates & Assignments \\
\hline
1 & 1/13 - 1/17 &  Initial Meeting\\
\hline
2 & 1/20 - 1/24 & Checkpoint 1.\\
\hline
3 & 1/27 - 1/30 &  \\
\hline
4 & 2/3 - 2/7 & Checkpoint 2.  \\
\hline
5 & 2/10 - 2/14 & \\
\hline
6 & 2/17 - 2/21 & Checkpoint 3. \\
\hline
7 & 2/24 - 2/28 & \\
\hline
8 & 3/3 - 3/7 & Checkpoint 5.  \\
\hline 
SPRING BREAK & 3/10 - 3/14& \\
\hline
9 & 3/17 - 3/21 &  \\
\hline
10 & 3/24 - 3/28 & Checkpoint 6. \\
\hline
11 & 3/31 - 4/4 &  \\
\hline
12 & 4/7 - 4/11 & Checkpoint 7. \\
\hline
13 & 4/14 - 4/18 &  EASTER BREAK (Friday).\\
\hline
14 & 4/21 - 4/25 & EASTER BREAK (Monday). Final Checkpoint. \\
\hline
15 & 4/28 - 5/2 & \textbf{Scholar's Day (4/29): Projects Due \& Presentations}. \\ 
\hline
16 & 5/5 - 5/7 & \\
\hline
Final's Week &  &  \\ 
\hline
\end{tabular}
\end{center}

\subsection{Checkpoints}

The class will meet for regular project checkpoints. At these checkpoints you will give a \textbf{five to ten minute presentation} that covers:
\begin{enumerate}
\item Intended project state based on current time line
\item Actual project state and progress since last checkpoint
\item Reflection on and evaluation of progress since the last checkpoint
\item \textit{Demonstration of Progress.} 
\item The plan for next checkpoint time period
\end{enumerate}
\textbf{Notice that you must actually demonstrate or present something concrete and/or functional at each checkpoint.}

\subsection{Attendance at Checkpoints}

Students must attend every checkpoint presentation and \textbf{must arrive on time}.  \textit{Tardiness to a checkpoint is more or less equivalent to a no show.}


\subsection{Scholar's Day Poster Presentations}

Students are expected to present their posters for at least one of the two poster session times on Scholar's Day.  This typically covers a 1-2 hour time block during the afternoon.  You should dress well and be prepared to engage passers by in conversations about your work.  This is not a formal presentation in front of a crowd.    

\subsection{Final Presentation}
 
The final presentation should be \textbf{30-45 minutes in length plus time for questions} and should address the following:
\begin{itemize}
\item Presentation and Demonstration of the Final Product
\item Reflection and Discussion of it's implementation. This can be, more or less, a summary of the checkpoint progress reflections with some big-picture items.
\item Technical and General lessons learned. (Should have both)
\end{itemize}
This presentation should not only highlight the work done but manner in which it was carried out.

\subsection{Course Engagement Expectations}

The weekly workload for this course will vary by student but on average should be about 5-7 hours per week.  While regular class meetings are scheduled for two hours a week, it is unlikely that we'll use all of that time each week.  We therefore expect students to dedicate at least 4-6 hours a week towards the development of their projects.  Being a capstone project, it is likely that your weekly work will exceed the expected amount.

\section{Grading}

At the completion of this course, the grade for both COMP401 and COMP402 is determined. Students will typically receive the same grade in both courses to reflect the work throughout the capstone project and not in one individual phase of the project. Grades will be determined based on the following items:
\begin{itemize}
\item Appropriateness of project difficulty (evaluated during COMP401) 
\item COMP401 checkpoints
\item COMP401 Technical Presentation
\item COMP401 Proposal Poster
\item Written Proposal
\item Proposal Presentation
\item COMP402 Checkpoint Presentations 
\item A Completed Project and required components
\item Final Poster and Presentation
\item Final Presentation 
\end{itemize}

More abstractly, what all of the above elements should reflect is a student's: 
\begin{itemize}
\item effective use of technical and problem solving skills befitting a major in Computer Science
\item professionalism
\item ability to make informed, mature decisions as they relate to a larger-scale project
\item understanding and appreciation of the computing disciplines
\end{itemize}


\end{document}

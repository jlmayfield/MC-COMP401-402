\documentclass[10pt]{article}
\usepackage{amsmath}
\usepackage{setspace}
\usepackage{hyperref}
\usepackage{booktabs}

\setlength{\textheight}{9in} \setlength{\topmargin}{-.5in}
\setlength{\textwidth}{6.5in} \setlength{\oddsidemargin}{0in}
\setlength{\evensidemargin}{0in}

\title{Syllabus \\ COMP402 Senior Project \\ Implementation}
\author{}
\date{ Spring 2017 }

\begin{document}
\maketitle

\section{Logistics}
\begin{itemize}
\item \textbf{Where: } Center for Science \& Business, Room 303
\item \textbf{When: } Th 11--11:50pm
\item \textbf{Instructors: }
\newline
\newline
\begin{tabular}{l}
\begin{tabular}{l}
\textbf{James \textit{Logan} Mayfield} \\
\textit{Office: }Center for Science and Business, Room 344 \\
\textit{Phone: } 309-457-2200 \\ %chktex 8
\textit{Email: } lmayfield at MONMOUTHCOLLEGE dot EDU \\
\textit{Website: } \url{http://jlmayfield.github.io} \\
\textit{Office Hours: } By appointment. See website for schedule.
\end{tabular}
\\
\newline \\
\begin{tabular}{l}
\textbf{Marta Tucker} \\
\textit{Office: } Center for Science and Business, Room 343  \\
\textit{Phone: } 309-457-2354 \\ \\ %chktex 8
\textit{Email: } marta at MONMOUTHCOLLEGE dot EDU \\
\textit{Office Hours: }  See posting by Office Door or by Appointment.
\end{tabular}
\end{tabular}
\item \textbf{Website: } \url{http://jlmayfield.github.io/teaching/COMP401-402/} \\ %chktex 8
\item \textbf{Credits: } $\dfrac{1}{2}$ course credit
\end{itemize}

\section{Course Content and Goals}

The senior project is the culminating experience of a student's major in Computer Science and draws upon everything the student has learned over the course of their studies. The project itself is a means to an end and not the ultimate goal of the capstone experience. Sufficiently interesting and complex projects rely on an abundance of existing research and fundamental principles of computing.  The project is, from this perspective, a concrete instantiation of these ideas and principles. By carrying out the project and presenting their work to technical and non-technical audiences students can demonstrate their understanding and mastery of some core element of the computing sciences as it appears in the real world context of their project.

Students are ultimately working towards pinpointing general, abstract, or theoretical concepts that support their project and clearly articulating how their work is a specific instance of these concepts. An important part of this process is the identification of seminal scholarly work that addresses the concept and its applications. Students too often attempt to reinvent the wheel in the course of their capstone work. Occasionally they're unaware that the wheel already exists. The emphasis on fundamentals is about identifying the wheels, the documents that lay out the general principles of those wheels, and focusing on applying those principles to the specific needs of the project. In doing so the student will better understand where their work sits in the broad spectrum of computing and can present it as such.

COMP 402 is focused on the implementation of the plans proposed by the student in COMP401 and the identification of the concrete instantiation of fundamental principles of computing at play within the various facets of the project. Each student in the class will give checkpoint presentations on a semi-regular basis in order to receive feedback from peers and faculty regarding the current state of student projects and their understanding of the project's underlying fundamentals.   At the end of the semester, students will use their project as the basis for a Scholar's Day poster and accompanying presentation.



\section{Attendance and Expectations}

Students in this course are expected to be respectful of their peers and the instructors. As this course is comprised entirely of student presentations, it is crucial that all students are always present and always on time.  Failure to arrive on time and be a productive member of the course will have a detrimental effect on the final grade and leaves a bad impression with faculty that are likely targets for job and graduate school recommendations.

\section{Course Deliverables}

The following elements of COMP402 contribute to the overall capstone grade:
\begin{itemize}
\item Checkpoint presentations
\item A Scholar's Day Poster
\item A Final Presentation
\item A publicly available, completed project
\item A bibliography of foundational work
\end{itemize}


\subsection{Checkpoints}

The class will meet for regular project checkpoints. At these checkpoints each student will give a \textbf{five to seven minute presentation} that covers:
\begin{enumerate}
\item The state of the project (1--2 minutes):
\begin{enumerate}
\item The expected state for this checkpoint based off the current time line
\item The actual state with a demonstration of progress
\item The expected state for the next checkpoint
\end{enumerate}
\item Computing Fundamentals (4--5 minutes)
\begin{enumerate}
\item Concepts, theories, and abstract principles
\item Seminal research and literature
\item Project specific instantiations
\end{enumerate}
\end{enumerate}

It is important to note that the bulk of each presentation is dedicated to the presentation of computing fundamentals as they appear in the project. The goal is to get accustomed to presenting the project in the context of a larger issue in computing as opposed to simply presenting the project.

\subsection{Scholar's Day Posters \& Final Presentations}

By the midpoint of the semester students will have identified at least one fundamental principle that acts as a cornerstone of their project. One of these principles will become the subject of the Scholar's Day poster and final presentation. It is important that students understand that \textit{the project is not the subject of the poster nor the presentation, it is the vehicle by which the actual subject is presented.}

The poster is to be done in the standard scientific research style. The CSB is ripe with examples of this kind of poster. Students will begin submitting drafts of their posters beginning shortly after midterm, and the final posters will be presented as a part of the Scholar's Day poster session.

The final presentation will be a 7--10 minute self-evaluation and debrief about your capstone experience. During this time you should briefly discuss the final state of your project, what you think went well this semester, what could have gone better, and what advice you might have for current and future 401/402 students as they begin working on their project.

\subsection{Completed Project}

By the end of the semester each student must have a completed project. It may entail everything proposed in COMP401 but it should be complete by some measurable sense of the word. For example, it may be a rough prototype of the proposed project or lack some of the originally proposed features, but still carries out some clearly identifiable part of the project as proposed. What's important is that it stands on its own and that the student presents it as such. Too often students lament the features they didn't get to and overemphasize what the project might have been as opposed to what it finally ended up being.

The final version of the project must be submitted to the instructors by Scholar's Day. In addition to submitting their work to the instructors, students must also find a means of making their work publicly available. Standard paths to this include, but are not limited to:
\begin{itemize}
\item Making the project Open-Source and hosting the code on a site like Github.
\item Making a completed application available on an app-store or some similar means of distribution.
\item Uploading a paper to a pre-print archive.
\item Submitting a paper for publication.
\item Hosing a website on a public server.
\end{itemize}
The exact means by which projects are made publicly accessible must be approved by the instructors ahead of time. Projects must be publicly accessible by Scholar's Day.

\subsection{Project Bibliography}

The poster and presentation are in depth demonstrations of the students ability to connect their work to a singular larger issues in computing. This bibliography is a demonstration of the student's ability to continue this practice across a broader range of issues as they relate to their the project. Over the course of COMP 401 and COMP 402 students should have accumulated several key texts related to the foundational computer science that underlies their project. The project bibliography should list and cite all of these sources. Some of these will be cited on the poster and presentation, some will not.  Multifaceted projects will rely on many different ideas and should thereby have a lengthier bibliography. More constrained projects might have shorter bibliographies as a reflection of their narrower focus. Either way, the bibliography should accurately reflect the breadth and scope of the project.

\section{Grading}

At the completion of this course, the grade for both COMP401 and COMP402 is determined. Students will typically receive the same grade in both courses to reflect the work throughout the capstone project and not in one individual phase of the project. Grades will be determined based on the following items:
\begin{itemize}
\item Appropriateness of project difficulty
\item COMP401 checkpoints
\item COMP401 Technical Presentation
\item COMP401 Proposal Poster
\item CMP401 Written Proposal
\item COMP401 Proposal Presentation
\item COMP402 Checkpoint Presentations
\item COMP 402 Scholar's Day Research Poster
\item COMP 402 Final Presentation
\item A Completed Project
\item A Project Bibliography
\end{itemize}

More abstractly, what all of the above elements should reflect is a student's:
\begin{itemize}
\item effective use of technical and problem solving skills befitting a major in Computer Science
\item professionalism
\item ability to make informed, mature decisions as they relate to a larger-scale project
\item understanding and appreciation of the computing disciplines
\end{itemize}

\section{Schedule}

Students will be split into two groups: group A and B. Group checkpoints will rotate every other week such that roughly half of the students will present every week.

\begin{center}
\begin{tabular}{lll}
Week & Dates & Assignments \\ \toprule
1 & 1/16--1/20 &  Initial Meeting.  \\
2 & 1/23--1/27 & Checkpoint 1A \\
3 & 1/30--2/3 & Checkpoint 1B \\
4 & 2/6--2/10 & Checkpoint 2A \\
5 & 2/13--2/17 & Checkpoint 2B \\
6 & 2/20--2/24 & Checkpoint 3A  \\
7 & 2/27--3/2 & Checkpoint 3B \\
 & 3/6--3/10 & \textit{Spring Break} \\
8 & 3/13--3/17  & Checkpoint 4A. Poster Draft. \\
9 & 3/20--3/24 & Checkpoint 4B. Poster Draft.   \\
10 & 3/27--3/31 & Checkpoint 5A. Poster Draft. \\
11 & 4/3--4/7 &  Checkpoint 5B. Poster Draft. \\
12 & 4/10--4/13 & Checkpoint 6A \& B --- Last Poster Draft. \\
13 & 4/18--4/21 & Posters Due. Bibliography Due.  \\
14 & 4/24--4/28 & Scholar's Day. Final Presentation. \\
15 & 5/1--5/3 &   \\ \midrule
  & 5/5 & Final's Time \textit{(3:00--6:00pm)}  \\
\end{tabular}
\end{center}

\subsection{Concurrent COMP401 Schedule}

This course is run concurrently with COMP401. Students enrolled in COMP402 must attend all the COMP401 checkpoints and vice versa. To avoid confusion, here is the other classes' schedule.

\begin{center}
\begin{tabular}{lll}
Week & Dates & Assignments \\ \toprule
1 & 1/16--1/20 &  Initial Meeting.  \\
2 & 1/23--1/27 & Project Ideas. \\
3 & 1/30--2/3 & Background: Project and Problem \\
4 & 2/6--2/10 & Background: Technical and Social \\
5 & 2/13--2/17 & Foundations \\
6 & 2/20--2/24 & Foundations  \\
7 & 2/27--3/2 & Tech-Talk Proposal \\
 & 3/6--3/10 & \textit{Spring Break} \\
8 & 3/13--3/17  & Tech-talk check-in \\
9 & 3/20--3/24 & Tech-Talk   \\
10 & 3/27--3/31 &  \\
11 & 4/3--4/7 &  Features and Specifications\\
12 & 4/10--4/13 & Plan and Timeline  \\
13 & 4/18--4/21 & \\
14 & 4/24--4/28 & Written Proposal Due. Presentation. \\
15 & 5/1--5/3 &   \\ \midrule
  & 5/5 & Final's Time \textit{(3:00--6:00pm)}  \\
\end{tabular}
\end{center}

\subsection{Course Engagement Expectations}

The weekly workload for this course will vary by student but on average should be about 5--7 hours per week.  While regular class meetings are scheduled for two hours a week, it is unlikely that we'll use all of that time each week.  We therefore expect students to dedicate at least 4--6 hours a week towards the development of their projects.  Being a capstone project, it is likely that your weekly work will exceed the expected amount.


\end{document}

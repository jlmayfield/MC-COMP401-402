\documentclass[10pt]{article}
\usepackage{amsmath}
\usepackage{setspace}
\usepackage{hyperref}
\usepackage{multirow}

\setlength{\textheight}{9in} \setlength{\topmargin}{-.5in}
\setlength{\textwidth}{6.5in} \setlength{\oddsidemargin}{0in}
\setlength{\evensidemargin}{0in}

\title{Syllabus - COMP402 - Senior Project - Implementation}
\author{James \textit{Logan} Mayfield}
\date{ Spring 2015 }

\begin{document}
\maketitle

\section{Logistics}
\begin{itemize}
\item \textbf{Where: } Center for Science \& Business, Room 309
\item \textbf{When: } Th 11-11:50 am 
\item \textbf{Instructors: }
\newline
\newline
\begin{tabular}{l}
\begin{tabular}{l}
\textbf{James \textit{Logan} Mayfield} \\
\textit{Office: }Center for Science and Business, Room 344 \\
\textit{Phone: } 309-457-2200 \\
\textit{Email: } lmayfield at MONMOUTHCOLLEGE dot EDU \\
\textit{Office Hours: } By Appointment.
\end{tabular}
\\
\newline \\
\begin{tabular}{l}
\textbf{Marta Tucker} \\
\textit{Office: } Center for Science and Business, Room 343  \\
\textit{Phone: } 309-457-2354 \\
\textit{Email: } marta at MONMOUTHCOLLEGE dot EDU \\
\textit{Office Hours: }  See posting by Office Door or by Appointment.
\end{tabular} 
\\
\newline \\
\begin{tabular}{l}
\textbf{Fabio Guerinoni} \\
\textit{Office: } Center for Science and Business, Room 342  \\
\textit{Phone: } 309-457-2247 \\
\textit{Email: } fguerinoni at MONMOUTHCOLLEGE dot EDU \\
\textit{Office Hours: }  See posting by Office Door or by Appointment.
\end{tabular} 
\end{tabular}
\item \textbf{Website: } \url{http://jlmayfield.github.io/MC-COMP401-402/}
\item \textbf{Credits: } $\dfrac{1}{2}$ course credit
\end{itemize}

\section{Course Content and Goals}

The senior project is designed allow students to demonstrate their abilities to apply all that they have learned about Computer Science and thereby act as a culminating experience for their studies in computing at Monmouth College.  Depending on class size, either groups of students or each individual student will be responsible for planning and carrying out a Computer Science related project.  

COMP 402 is focused on implementation of the plans proposed by the student in COMP401.  The class will meet on a semi-regular basis for  brief presentations on the current state of the projects in order to receive feed back from peers and faculty.  At the end of the semester, students will present the final results of their work to the campus at large and at the Science Poster session held each spring on Scholar's Day.

\section{Attendance and Expectations}

Students in this course are expected to be respectful of their peers and the instructor. As this course is comprised entirely of student presentations, it is crucial that you are always in class and always on time.  Come prepared to listen, discuss, and present.  Failure to arrive on time and be a productive member of the course will have a detrimental effect on your grade.  Not only that, but it leaves a certain impression on the faculty from whom you'll soon want letters of recommendation for jobs.  


\section{Deliverables}

Students projects generally fall into one of two categories: Software development based projects or research projects.  Development projects are generally programming centric and result in working software.  Research projects fall in line with traditional scientific research projects and generally result in a paper. 

\subsection{Software Development Based Projects}

Software-based projects must include:
\begin{itemize}
\item A working, publicly available final product
\item Source code with relevant documentation
\item End-User documentation
\item A research-style poster
\end{itemize}
Your project does not need to be open-source but you must submit your source to the course instructors.

Software may be made publicly available in many different ways.  A few options include:
\begin{itemize}
\item Hosting the code and executable as an open-source project on a site such as \url{http://github.com}
\item Hosting final product, possibly closed-source, on software hosting/download site such as an App Store.
\end{itemize}
\textit{Students may ``publish'' their work in other ways, but must get the OK from the instructors before doing so.}

\subsection{Research Projects}

Research projects must include:
\begin{itemize}
\item A complete, published paper 
\item An annotated bibliography and works cited
\item A research-style poster
\end{itemize}

Publication of final papers need not be in a peer-reviewed journal or conference proceeding.  The following are a few examples of ways in which students might meet their publication requirements:
\begin{itemize}
\item Submission to a peer-reviewed journal or conference with no requirement for acceptance
\item Submission to a reputable pre-print archive such as \url{http://www.arxiv.org}
\end{itemize}
\textit{Students may ``publish'' their work in other ways, but must get the OK from the instructors before doing so.}

\subsection{Checkpoints}

The class will meet for regular project checkpoints. At the start of the semester the projects will be split into two groups and those groups will alternate weeks for presentations. At these checkpoints you will give a \textbf{five to ten minute presentation} that covers:
\begin{enumerate}
\item Intended project state based on current time line
\item Actual project state and progress since last checkpoint
\item Reflection on and evaluation of progress since the last checkpoint
\item \textit{Demonstration of Progress.} 
\item The plan for next checkpoint time period
\end{enumerate}
\textbf{Notice that you must actually demonstrate or present something concrete and/or functional at each checkpoint.} 

At your first checkpoint, you must present your time line with respect to the remaining checkpoint presentations. You should be able to modify the time line from your proposal to fit the checkpoint schedule listed below.


\subsection{Final Presentation}
 
The final presentation should be \textbf{30-45 minutes in length plus time for questions} and should address the following:
\begin{itemize}
\item High-level overview of the project and its goals
\item Presentation and demonstration of the final product
\item Presentation of a key technical problem overcome within the context of the project.
\item Technical and General lessons learned. (Should have both)
\end{itemize}
This presentation should not only highlight the work done but manner in which it was carried out.  A key moment in the presentation is identifying a key, general problem in Computer Science that your encountered while completing your project.  You should discuss the specific form that problem took in the context of your project and how your solution relates to known solutions and best practices. In doing this, you'll demonstrate an awareness of the larger body of work in computer science as it relates to your project.

\subsection{Scholar's Day Posters}

Everyone will produce a research poster that complements your final presentation. The poster will focus on the key, general problem in CS that you intend to highlight in your presentation.  In short, your poster presents a very specific instance of a known problem in computer science and the solution you developed in the context of your project. 

Students can expected to present their posters for at least one of the two poster session times on Scholar's Day.  This typically covers a 1-2 hour time block during the afternoon.  You should dress well and be prepared to engage passers by in conversations about your work.  This is not a formal presentation in front of a crowd.    

The poster should be 


\section{Grading}

At the completion of this course, the grade for both COMP401 and COMP402 is determined. Students will typically receive the same grade in both courses to reflect the work throughout the capstone project and not in one individual phase of the project. Grades will be determined based on the following items:
\begin{itemize}
\item Appropriateness of project difficulty (evaluated during COMP401) 
\item COMP401 checkpoints
\item COMP401 Technical Presentation
\item COMP401 Proposal Poster
\item CMP401 Written Proposal
\item COMP401 Proposal Presentation
\item COMP402 Checkpoint Presentations 
\item COMP 402 Research Poster and Scholar's Day Participation 
\item COMP 402 Final Presentation 
\item A Completed Project and required components
\end{itemize}

More abstractly, what all of the above elements should reflect is a student's: 
\begin{itemize}
\item effective use of technical and problem solving skills befitting a major in Computer Science
\item professionalism
\item ability to make informed, mature decisions as they relate to a larger-scale project
\item understanding and appreciation of the computing disciplines
\end{itemize}

\section{Schedule}

\begin{center}
\begin{tabular}{|c|c|r|}
\hline 
Week & Dates & Assignments \\
\hline
1 & 8/25 - 8/28 &  \\
\hline 
2 & 8/31 - 9/4 &   \\
\hline
3 & 9/7 - 9/11 &   \\
\hline
4 & 9/14 - 9/18 &   \\
\hline
5 & 9/21 - 9/25 &  \\
\hline
6 & 9/28 - 10/2 & \\
\hline
7 & 10/5 - 10/9  &  \\
\hline 
8 & 10/12 - 10/15 &  FALL BREAK (F) \\
\hline
9 & 10/21 - 10/23 & FALL BREAK (M,Tu) \\
\hline
10 & 10/26 - 10/30 &  \\
\hline
11 & 11/2 - 11/6 & \\
\hline
12 & 11/9 - 11/13 &  \\
\hline
13 & 11/16 - 11/20 & \\
\hline
14 & 11/23 - 11/24 &  THANKSGIVING BREAK (W-F).   \\
\hline
15 & 11/30 - 12/4 & \\ 
\hline
16 & 12/7 - 12/9 &   Reading Day (Th). \\
\hline
Final's Week & 12/12 (6:30-9:30pm) & Final Exam. \\ 
\hline
\end{tabular}
\end{center}

\subsection{Course Engagement Expectations}

The weekly workload for this course will vary by student but on average should be about 5-7 hours per week.  While regular class meetings are scheduled for two hours a week, it is unlikely that we'll use all of that time each week.  We therefore expect students to dedicate at least 4-6 hours a week towards the development of their projects.  Being a capstone project, it is likely that your weekly work will exceed the expected amount.


\end{document}

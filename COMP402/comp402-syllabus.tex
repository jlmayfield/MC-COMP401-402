\documentclass[10pt]{article}
\usepackage{amsmath}
\usepackage{setspace}
\usepackage{hyperref}
\usepackage{multirow}

\setlength{\textheight}{9in} \setlength{\topmargin}{-.5in}
\setlength{\textwidth}{6.5in} \setlength{\oddsidemargin}{0in}
\setlength{\evensidemargin}{0in}

\title{Syllabus - COMP402 - Senior Project - Implementation}
\author{James \textit{Logan} Mayfield}
\date{ Spring 2016 }

\begin{document}
\maketitle

\section{Logistics}
\begin{itemize}
\item \textbf{Where: } Center for Science \& Business, Room 303
\item \textbf{When: } Th 12-12:50pm 
\item \textbf{Instructors: }
\newline
\newline
\begin{tabular}{l}
\begin{tabular}{l}
\textbf{James \textit{Logan} Mayfield} \\
\textit{Office: }Center for Science and Business, Room 344 \\
\textit{Phone: } 309-457-2200 \\
\textit{Email: } lmayfield at MONMOUTHCOLLEGE dot EDU \\
\textit{Office Hours: } By Appointment.
\end{tabular}
\\
\newline \\
\begin{tabular}{l}
\textbf{Marta Tucker} \\
\textit{Office: } Center for Science and Business, Room 343  \\
\textit{Phone: } 309-457-2354 \\
\textit{Email: } marta at MONMOUTHCOLLEGE dot EDU \\
\textit{Office Hours: }  See posting by Office Door or by Appointment.
\end{tabular} 
\end{tabular}
\item \textbf{Website: } \url{http://jlmayfield.github.io/MC-COMP401-402/}
\item \textbf{Credits: } $\dfrac{1}{2}$ course credit
\end{itemize}

\section{Course Content and Goals}

The senior project is designed to be the culminating experience of a Computer Science major program.  It calls upon students to draw on everything they have learned over the course of their studies. Students work in either small groups individually to plan and carry out a major research or development project.  

The project itself is a means to an end and not the ultimate goal of the capstone experience. Sufficiently interesting and complex projects rely on an abundance of existing research and fundamental principals of computing.  What we want is for you to identify and understand this context, learn to work in this context by way of completing your project, and ultimately to communicate these ideas to technical and non-technical people by using the specifics of your project as the concrete instantiation of this context. 

COMP 402 is focused on the implementation of the plans proposed by the student in COMP401.  The class will meet on a semi-regular basis for  brief presentations on the current state of the projects in order to receive feedback from peers and faculty.  At the end of the semester, students will present the final results of their work to the campus at large and at the Science Poster session held each spring on Scholar's Day and at public, final presentations. 

Recall from 401 that the overall goals of the capstone are that:
\begin{itemize}
\item Students will carry out a major computer science research or development project as proposed in COMP401
\item Students will develop a firm understanding of the community of researchers and developers in which their work places them
\item Students will develop a firm understanding of the impact that their current and future work may  have on society at large 
\end{itemize}
Your final poster  and presentation are less about the first item and more about how the first item relates to the second two items.  We want you to help the Monmouth College community at large understand your work, its merit, and its importance. If you do that successfully, then you will have been successful in your capstone. 


\section{Attendance and Expectations}

Students in this course are expected to be respectful of their peers and the instructor. As this course is comprised entirely of student presentations, it is crucial that you are always in class and always on time.  Come prepared to listen, discuss, and present.  Failure to arrive on time and be a productive member of the course will have a detrimental effect on your grade.  Not only that, but it leaves a certain impression on the faculty from whom you'll soon want letters of recommendation for jobs.  

\section{Deliverables}

Students projects generally fall into one of two categories: Software development based projects or research projects.  Development projects are generally programming centric and result in working software.  Research projects fall in line with traditional scientific research projects and generally result in a paper. 

\subsection{Software Development Based Projects}

Software-based projects must include:
\begin{itemize}
\item A working, publicly available final product
\item Source code with relevant documentation
\end{itemize}
Your project does not need to be open-source but you must submit your source to the course instructors.

Software may be made publicly available in many different ways.  A few options include:
\begin{itemize}
\item Hosting the code and executable as an open-source project on a site such as \url{http://github.com}
\item Hosting final product, possibly closed-source, on software hosting/download site such as an App Store.
\end{itemize}
\textit{Students may ``publish'' their work in other ways, but must get the OK from the instructors before doing so.}

\subsection{Research Projects}

Research projects must include:
\begin{itemize}
\item A complete, publicly available paper 
\item An annotated bibliography and works cited
\end{itemize}

Publication of final papers need not be in a peer-reviewed journal or conference proceeding.  The following are a few examples of ways in which students might meet their publication requirements:
\begin{itemize}
\item Submission to a peer-reviewed journal or conference with no requirement for acceptance
\item Submission to a reputable pre-print archive such as \url{http://www.arxiv.org}
\end{itemize}
\textit{Students may ``publish'' their work in other ways, but must get the OK from the instructors before doing so.}


\subsection{Checkpoints}

The class will meet for regular project checkpoints. At these checkpoints you will give a \textbf{five to seven minute presentation} that covers:
\begin{enumerate}
\item Intended project state based on current time line
\item Actual project state and progress since last checkpoint
\item Reflection on and evaluation of progress since the last checkpoint
\item \textit{Demonstration of Progress.} 
\item The plan for next checkpoint time period
\end{enumerate}
\textbf{Notice that you must actually demonstrate or present something concrete and/or functional at each checkpoint.} 

At your first checkpoint, you must present your time line with respect to the remaining checkpoint presentations. You should be able to modify the time line from your proposal to fit the checkpoint schedule listed below.


\subsection{Final Presentation}
 
The final presentation should be \textbf{30-45 minutes in length plus time for questions} and should address your project in the context of one critical element of the foundational computer science research underlying your work. Thus, the topic of your presentation is not your project per se but the foundational computer science and its exact role in your work. Your project and the problem it addresses provides a concrete real-world context in which the foundations of computer science can be presented and discussed.  Your primary goal for this presentation is to do just this for a general audience. 
 

\subsection{Scholar's Day Posters}

Everyone will produce a research poster that complements your final presentation. The poster is essential the presentation condensed to the poster format. When applicable, this poster will be presented at Scholar's Day in the spring. If this is not possible, then other arrangements will be made for the presentation of the poster. Regardless, the poster should be made with Scholar's Day in mind. 

\section{Grading}

At the completion of this course, the grade for both COMP401 and COMP402 is determined. Students will typically receive the same grade in both courses to reflect the work throughout the capstone project and not in one individual phase of the project. Grades will be determined based on the following items:
\begin{itemize}
\item Appropriateness of project difficulty (evaluated during COMP401) 
\item COMP401 checkpoints
\item COMP401 Technical Presentation
\item COMP401 Proposal Poster
\item CMP401 Written Proposal
\item COMP401 Proposal Presentation
\item COMP402 Checkpoint Presentations 
\item COMP 402 Research Poster and Scholar's Day Participation 
\item COMP 402 Final Presentation 
\item A Completed Project and required components
\end{itemize}

More abstractly, what all of the above elements should reflect is a student's: 
\begin{itemize}
\item effective use of technical and problem solving skills befitting a major in Computer Science
\item professionalism
\item ability to make informed, mature decisions as they relate to a larger-scale project
\item understanding and appreciation of the computing disciplines
\end{itemize}

\section{Schedule}


\begin{center}
\begin{tabular}{|c|c|r|}
\hline 
Week & Dates & Assignments \\
\hline
1 & 1/11 - 1/15 &  First Meeting\\
\hline
2 & 1/18 - 1/22 & Checkpoint 1 \\
\hline
3 & 1/25 - 1/29 &   \\
\hline
4 & 2/1 - 2/5 & Checkpoint 2  \\
\hline
5 & 2/8 - 2/12 &  \\
\hline
6 & 2/15 - 2/19 & Checkpoint 3 \\
\hline
7 & 2/22 - 2/26 &   \\
\hline
8 & 2/29 - 3/4 & Checkpoint 4  \\
\hline 
SPRING BREAK & 3/7 - 3/11 &  \\
\hline
9 & 3/14 - 3/18 &  \\
\hline
10 EASTER BREAK (Fr)& 3/21 - 3/24 & Checkpoint 5. Poster Draft Due.\\
\hline
11 EASTER BREAK (Mo)& 3/29 - 4/1 &  \\
\hline
12 & 4/4 - 4/8 & Checkpoint 6. Poster Draft Due.\\
\hline
13 & 4/11 - 4/15 &   \\
\hline
14 & 4/18 - 4/22 &  Checkpoint 7. Poster Due. \\
\hline
15 FOUNDER'S DAY (Tu) & 4/25 - 4/29 & Poster Session \& Final Presentations. \\ 
\hline
16 & 5/2 - 5/4 & \\
\hline
Final's Week & 5/6 (3:00pm) & Exit Interviews \\ 
\hline
\end{tabular}
\end{center}

\subsection{Course Engagement Expectations}

The weekly workload for this course will vary by student but on average should be about 5-7 hours per week.  While regular class meetings are scheduled for two hours a week, it is unlikely that we'll use all of that time each week.  We therefore expect students to dedicate at least 4-6 hours a week towards the development of their projects.  Being a capstone project, it is likely that your weekly work will exceed the expected amount.


\end{document}

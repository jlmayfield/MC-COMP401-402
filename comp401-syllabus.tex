\documentclass[10pt]{article}
\usepackage{amsmath}
\usepackage{setspace}
%\usepackage{multirow}

\setlength{\textheight}{9in} \setlength{\topmargin}{-.5in}
\setlength{\textwidth}{6.5in} \setlength{\oddsidemargin}{0in}
\setlength{\evensidemargin}{0in}

\title{Syllabus - COMP401 - Senior Project - Research}
\author{}
\date{Fall 2014}

\begin{document}
\maketitle

\section{Logistics}
\begin{itemize}
\item \textbf{Where: } Center for Science and Business, Room 277
\item \textbf{When: } Th, 11-11:50
\item \textbf{Instructors: }
\newline
\newline
\begin{tabular}{l}
\begin{tabular}{l}
\textbf{James \textit{Logan} Mayfield} \\
\textit{Office: }Center for Science and Business, Room 344 \\
\textit{Phone: } 309-457-2200 \\
\textit{Email: } lmayfield at MONMOUTHCOLLEGE dot EDU \\
\textit{Office Hours: } By Appointment.
\end{tabular}
\\
\newline \\
\begin{tabular}{l}
\textbf{Marta Tucker} \\
\textit{Office: } Center for Science and Business, Room 343  \\
\textit{Phone: } 309-457-2354 \\
\textit{Email: } marta at MONMOUTHCOLLEGE dot EDU \\
\textit{Office Hours: }  See posting by Office Door or by Appointment.
\end{tabular} 
\\
\newline \\
\begin{tabular}{l}
\textbf{Fabio Guerinoni} \\
\textit{Office: } Center for Science and Business, Room 342  \\
\textit{Phone: } 309-457-2247 \\
\textit{Email: } fguerinoni at MONMOUTHCOLLEGE dot EDU \\
\textit{Office Hours: }  See posting by Office Door or by Appointment.
\end{tabular} 
\end{tabular}

\item \textbf{Credits: } $\dfrac{1}{2}$ course credit
\end{itemize}


\section{Course Content and Goals}

The senior project is designed to be the culminating experience of a Computer Science major program.  It calls upon students to draw on everything they have learned over the course of their studies. Students work in either small groups individually to plan and carry out a major research or development project.  

COMP 401 is focused on developing a detailed proposal for the senior project. Students will take the semester to research topics surrounding their project and prepare themselves to immediately begin implementing their proposal the following spring.  Through out the semester, students will make regular checkpoint presentations demonstrating their progress.  At the end of the semester, students well present their proposed project to a general audience as well as prepare a poster highlighting the work they plan to carry out in the spring. 

\subsection{Course Goals}
\begin{itemize}
\item Students will plan a major computer science research or development project
\item Students will prepare to undertake their project when they return in the spring by gaining working knowledge of the technology and previous work upon which their project relies
\item Students will begin to form a understanding of the community of researchers and developers in which their work places them
\item Students will begin to form a understanding of the impact that their current and future work will have on society at large 
\end{itemize}

\section{Expectations and Policies}

Students in this course are expected to be respectful of their peers and the instructor. If you're unsure what that might mean, then consider the following guidelines. 
\begin{itemize}
\item Be respectful of others.  Don't create unnecessary distractions.  Turn cell phones off, on silent or leave them in the dorm.  Class time is not the time for checking email, surfing the web and IMing.  \textit{Come to class ready and interested in learning and if you're not, don't behave in such a way that prevents others from doing so.}
\item You're in college.  College is meant to provide an education.  Therefore, you are, for all intents and purposes, a \textit{professional student}.  Your work should reflect a solid level of professionalism and be neat and orderly.  Take the extra time to make it presentable.  Crumpled papers with various liquid stains on them are not presentable.  Think of the instructor as your boss and that the quality of your paycheck depends on the quality of the work.  \textit{You don't have to always love the work you do, but you should always do it to the best of your capabilities.}
\item Attending class is not by itself sufficient for learning the material.  You're expected to read the sections of the text as they are covered in class.  You are encouraged to go beyond the material.  Make use of available resources such as tutors and the high availability of your instructor.  \textit{Don't expect to get an A just by showing up and doing the least amount of work that you can.}
\end{itemize}

There are several \textit{strict policies} that result from students continually failing to live up to these expectations.  \textit{You are responsible for understanding and abiding by these rules.}
\begin{itemize}
\item \textit{Late Assignments: }In general, late assignments will \textit{not} be accepted.  If you feel you have a justified reason for the assignment being late you may set up an appointment to meet with the instructor and plead your case.  Situations beyond your control are understandable and exceptions can and will be made.
\item \textit{Attendance: }You're an adult, you can choose to not come to class, but if you're regularly absent or late then expect it to effect your participation grade.  If you do miss/skip class \textit{you are still responsible for everything covered on that day}.  If you have no valid reason for missing class, do not expect the instructor to spend the time to re-present the class to you individually.   
\item \textit{Participation: }  Cellphone usage in class is not allowed, this includes text messages.  Turn off the ringers or leave them at home.  Computer usage is limited to activities in support of the course.  This does not include IMs, Facebook, checking email, general web surfing, poker, fantasy sports leagues, forum trolling, mine sweeper, etc.  This behavior is rude and can be a real distraction to others.  Repeated failure to abide by this policy will have a negative effect on your grade.  
\item \textit{Quality of Work:} There are several minimal requirements that your assignments must meet.
\begin{itemize}
\item \textit{Staples - } Assignments that take up more than one page must be stapled.  Unstapled assignments will either be returned to you to be stabled ASAP or points will be deducted.  
\item \textit{Neatness - }  Make every attempt to make your work neat and orderly:  label problems, avoid excessive scratching out of mistakes (use pencil if you are prone to errors) and if you use spiral bound paper tear off the edges. 
\item \textit{Show Work - } Rarely are answers alone sufficient for full credit.  Show your work whenever prudent.  If you're unsure if work is needed, \textit{ask!}
\end{itemize}
\end{itemize}

\subsection{Collaboration}

In general, you are encouraged to make use of the resources available to you.  This means it is OK to seek help from a friend, tutor, instructor, internet, etc.  However, \textit{copying of answers and any act worthy of the label of ``cheating'' is never permissible!}  It is understandable that when you work with a partner or a group that the resultant product is often extremely similar.  This is acceptable but be prepared to be asked to defend your collaborations to the instructor.  \textit{You should always be able to reproduce an answer on your own, and if you cannot you likely \textbf{do not really known the material.}} 
\begin{itemize}
\item When assignments are meant to be done in groups, you will be directed to turn in one set of solutions per group.
\item All other assignments should represent your own work and effort.
\end{itemize}
All of the Monmouth College rules on academic dishonesty apply.  If you violate the rules be prepared to face the consequences of your actions. 



\section{Grades}

This courses adheres to the Mathematics and Computer Science grade scale.  Assignments and final grades will not be curved except when deemed necessary by the instruction.  Percentage grades translate to letter grades as follows:
\newline
\begin{small}
\begin{tabular}{lc}
94-100 & A \\
90-93 & A- \\
88-89 & B+ \\
82-87 & B \\
79-81 & B- \\
76-78 & C+ \\
70-75 & C \\
67-69 & C- \\
64-66 & D+ \\
58-63 & D \\
55-57 & D- \\
0-54 & F 
\end{tabular}
\end{small}
\newline
You are always welcome to challenge a grade that you feel is unfair or calculated incorrectly.  Mistakes made in your favor will never be corrected to lower your grade.  Mistakes made not in your favor will be corrected.  \textit{Basically, after the initial grading your score can only go up as the result of a challenge.}

\subsection{Grading}

Upon the completion of COMP402 in the spring, your grade for both COMP401 and COMP402 is determined. Students will typically receive the same grade in both courses to reflect the work throughout the capstone project and not in one individual phase of the project. Grades will be determined based on the following items:
\begin{itemize}
\item \textit{Appropriateness of project difficulty (evaluated during COMP401) }
\item \textit{COMP401 checkpoints}
\item \textit{COMP401 Technical Presentation}
\item \textit{COMP401 Proposal Poster}
\item \textit{Written Proposal}
\item \textit{Proposal Presentation}
\item COMP402 Checkpoint Presentations 
\item A Completed Project and required components
\item Final Poster and Presentation
\item Final Presentation 
\end{itemize}
Italicized items are specific to COMP 401. Exact weights are not assigned to the above items to allow students the ability to recover from poor performance on one or more items by strong performances in other areas of the project.  Students will receive feedback on all the above items including an indication of how it is effecting their grade.  If at anytime the student has a question about their grade, they should seek out one or both of the instructors. 

Ultimately, the student is being evaluated on the following criteria:
\begin{itemize}
\item effective use of technical and problem solving skills befitting a major in Computer Science
\item professionalism
\item ability to make informed, mature decisions as they relate to a larger-scale project
\item understanding and appreciation of the computing disciplines
\end{itemize}



\subsection{Course Engagement Expectations}

The weekly workload for this course will vary by student but on average should be about 5-7 hours per week.  While regular class meetings are scheduled for two hours a week, it is unlikely that we'll use all of that time each week.  We therefore expect students to dedicate at least 4-6 hours a week towards the development of their project proposals.  This time can include research, preliminary coding, writing, meeting with professors, and so forth.


\section{Checkpoint Calendar}

Checkpoint presentations will occur on roughly a weekly basis. \textit{This calendar is subject to change based on the circumstances of the course.}
\begin{center}
\begin{tabular}{|c|c|r|}
\hline 
Week & Dates & Assignments \\
\hline
1 & 8/25 - 8/29 &  \\
\hline 
2 & 9/1 - 9/5 &  Initial Concepts and Ideas \\
\hline
3 & 9/8 - 9/12 & Initial Project Proposal   \\
\hline
4 & 9/15 - 9/19 & Community \\
\hline
5 & 9/22 - 9/26 & Background and History \\
\hline
6 & 9/29 - 10/1 & Stakeholder. FALL BREAK (F). \\
\hline
7 & 10/8 - 10-10  & FALL BREAK (M,Tu). Tools \& Bibliography   \\
\hline 
8 & 10/14 - 10/18 & Features \& Key Research \\
\hline
9 & 10/20 - 10/24 & Mentoring Day (W). Pre-Tech Talk.\\
\hline
10 & 10/27 - 10/31 & Technical Talk \\
\hline
11 & 11/3 - 11/7 & Preliminary Design\\
\hline
12 & 11/10 - 11/14 & Final Design \& Plan \\
\hline
13 & 11/17 - 11/21 & Pre-Proposal Wrap-Up\\
\hline
14 & 11/24 - 11/25 &  THANKSGIVING BREAK (W-F).   \\
\hline
15 & 12/1- 12/4 & \textbf{Final Proposal Presentations}\\ 
\hline
\end{tabular}
\end{center}



\end{document}
